
\documentclass[12pt,letter]{article}

%% \ifCLASSOPTIONcompsoc
%% % IEEE Computer Society needs nocompress option
%% % requires cite.sty v4.0 or later (November 2003)
%% \usepackage[nocompress]{cite}
%% \else
%% % normal IEEE
%% \usepackage{cite}
%% \fi

%% \usepackage[fleqn]{amsmath}
\usepackage[margin=1.0in]{geometry}
\usepackage{amsmath,amsfonts,amsthm,bm}
\usepackage{breqn}
\usepackage{amsmath}
\usepackage{amssymb}
\usepackage{tikz}
\usepackage{algorithm2e}
\usepackage{siunitx}
\usepackage{graphicx}
\usepackage{subcaption}
%% \usepackage{datetime}
\usepackage{multirow}
\usepackage{multicol}
\usepackage{mathrsfs}
\usepackage{fancyhdr}
\usepackage{fancyvrb}

\pagestyle{fancy}

\usetikzlibrary{arrows}

\DeclareMathOperator*{\argmin}{argmin}
\newcommand*{\argminl}{\argmin\limits}

\newcommand{\mathleft}{\@fleqntrue\@mathmargin0pt}
\newcommand{\R}{\mathbb{R}}
\newcommand{\Z}{\mathbb{Z}}
\newcommand{\N}{\mathbb{N}}

\begin {document}

\rhead{(Bill) Yuan Liu, student \#: 996954078\\ Date: 2019/10/12}
\lhead{ECE1647F - Nonlinear Systems - Assignment 2}

% \begin{align*}    
% \end{align*}

\begin{itemize}
  
\item 1.5\\
  From the graph of x vs. t, it is true that, \\
  $(\exists t_A)(\exists t_B)(t_A\neq t_B) \wedge (x(t_A,x_0)=x(t_B,x_0)) \wedge (\frac{\partial}{\partial t}x(t_A,x_0) \neq \frac{\partial}{\partial t}x(t_B,x_0))$,\\
  which occurs in the inflection region of the graph. Thus, the vector field is time varying ($f(t,x)$). x(t) cannot be a solution of a scalar differential equation with locally Lipschitz autonomous vector field.
  
\item 1.10\\
  
  let $\chi=\begin{bmatrix}t \\ x\end{bmatrix} \in \R \times \mathcal{X}$\\

  let $\bar{f}(\chi)=\begin{bmatrix}1 \\ f(\chi)\end{bmatrix} = \dot{\chi}$\\

  $f(\chi)=f(\begin{bmatrix}t \\ x\end{bmatrix})$\\

  if map $(t,x)\mapsto f(t,x)$ is locally Lipschitz,\\
  show $\phi(t_0,t_0,x_0)=x_0$\\

  $(t,x)\mapsto f(t,x)$ is locally Lipschitz:\\
  $(\forall \chi)(\exists L)\|f(\chi)-f(\chi_0)\| \leq L \|\chi-\chi_0\|$ for each $\chi_0$\\
  
  $\|\bar{f}(\chi)-\bar{f}(\chi_0)\|=\left\|\begin{bmatrix}1 \\ f(\chi)\end{bmatrix}-\begin{bmatrix}1 \\ f(\chi_0)\end{bmatrix}\right\|=\left\|\begin{bmatrix}0 \\ f(\chi)-f(\chi_0)\end{bmatrix}\right\|$\\
  $\|\bar{f}(\chi)-\bar{f}(\chi_0)\|=\left\|f(\chi)-f(\chi_0)\right\|$\\
  $(\forall \chi)(\exists L)\|\bar{f}(\chi)-\bar{f}(\chi_0)\| \leq L \|\chi-\chi_0\|$ for each $\chi_0$\\

  $\bar{f}$ is locally Lipschitz\\

  let $\bar{\phi}: \R \times (\R \times \mathcal{X}) \rightarrow \R \times \mathcal{X}$\\
  $\phi = \R \times \R \times \mathcal{X} \rightarrow \mathcal{X}$\\
  $\phi = \text{project}_x \bar{\phi}$, where $\text{project}_x \bar{\phi}: \R \times \mathcal{X} \rightarrow \mathcal{X},\ \text{project}_x \bar{\phi}=\text{map } \begin{bmatrix}t \\ x\end{bmatrix} \mapsto x$\\

  use theorem 1.20: if $\bar{f}$ is locally Lipschitz on $\R \times \mathcal{X}$ then, $(\forall \chi_0)(\exists \chi(t))$ $\chi(t)$ is unique and maximal with $\chi(t_0)=\chi_0=\begin{bmatrix}t_0\\x_0\end{bmatrix}$\\
  
  let $\bar\phi(t,\chi_0)$ be that solution satisfying initial condition and $\bar{f}$\\

  $\phi(\tau,t_0,x_0)=project_x(\bar{\phi}(\tau,\chi_0))$\\

  $\phi(t_0,t_0,x_0)=project_x(\bar{\phi}(t_0,\chi_0))=project_x(\chi_0)=project_x(\begin{bmatrix}t_0\\x_0\end{bmatrix})=x_0$\\

  show $\partial_{t'} \phi(t',t_0,x_0)=f(t',\phi(t',t_0,x_0))$\\

  $\partial_{t'} \phi(t',t_0,x_0)=\frac{\partial}{\partial t'}\left(project_x \int_{0}^{t'}\bar{f}(\chi(\tau))d\tau + \chi_0\right)$\\

  $\partial_{t'} \phi(t',t_0,x_0)=project_x \bar{f}(\chi(t'))=project_x\begin{bmatrix}1 \\ f(\chi(t'))\end{bmatrix}\Big|_{\chi(0)=\chi_0}$\\

  $\partial_{t'} \phi(t',t_0,x_0)=f(\chi(t'))=f\left(\begin{bmatrix}t(t') \\ x(t')\end{bmatrix}\right)\Big|_{\chi(0)=\chi_0}$\\

  $t(t')=\int_0^{t'} 1 d\tau = t'+t_0=t'$\\
  
  $\partial_{t'} \phi(t',t_0,x_0)=f\left(\begin{bmatrix}t' \\ \phi(t',\chi_0)\end{bmatrix}\right)=f\left(\begin{bmatrix}t' \\ \phi(t', [t_0, x_0])\end{bmatrix}\right) = f(t',\phi(t', t_0, x_0))$\\

  show $(t,t_0,x_0) \mapsto \phi(t,t_0,x_0)$ is continuous\\

  Using theorem 1.26: $\bar{f}$ is locally Lipschitz on $X=\R \times \mathcal{X}$, $W=\{(t,\chi_0)\} \in \R \times (\R \times \mathcal{X}), t \in T_{\chi_0}$:=maximal interval of existence, $\bar{\phi}(t,\chi_0)$ being maximal solution with initial condition $\chi_0$, then W is open and $(t,\chi_0) \mapsto \bar{\phi}(t,\chi_0)$ is continuous. Then, $project_x \bar{\phi}=\phi$, and $(t,t_0,x_0) \mapsto \phi(t,t_0,x_0)$ is continuous. 
  
  \pagebreak
\item 1.11\\

  extending previous question to include constants $\lambda \in \R^m$\\
  
  let $\chi=\begin{bmatrix}t \\ x \\ \lambda \end{bmatrix} \in \R \times \mathcal{X} \times \R^m$\\
    
  let $\bar{f}(\chi)=\begin{bmatrix}1 \\ f(\chi) \\ 0^{m \times 1} \end{bmatrix} = \dot{\chi}$\\

  $f(\chi)=f\left(\begin{bmatrix}t \\ x \\ 0^{m \times 1} \end{bmatrix}\right)$\\

  if map $(t,x,\lambda)\mapsto f(t,x,\lambda)$ is locally Lipschitz,\\
  show $\phi(t,t_0,x_0,\lambda_0)$ is continuous satisfying $\phi(t_0,t_0,x_0,\lambda_0)=x_0$\\

  since map $(t,x,\lambda)\mapsto f(t,x,\lambda)$ is locally Lipschitz:\\
  $(\forall \chi)(\exists L)\|f(\chi)-f(\chi_0)\| \leq L \|\chi-\chi_0\|$ for each $\chi_0$\\
  
  $\|\bar{f}(\chi)-\bar{f}(\chi_0)\|=\left\|\begin{bmatrix}1 \\ f(\chi) \\ 0^{m \times 1}\end{bmatrix}-\begin{bmatrix}1 \\ f(\chi_0) \\ 0^{m \times 1} \end{bmatrix}\right\|$\\

  $\|\bar{f}(\chi)-\bar{f}(\chi_0)\|=\left\|\begin{bmatrix}0 \\ f(\chi)-f(\chi_0) \\ 0^{m \times 1} \end{bmatrix}\right\|=\left\|\begin{bmatrix}f(\chi)-f(\chi_0)\end{bmatrix}\right\|$\\
  
  $(\forall \chi)(\exists L)\|\bar{f}(\chi)-\bar{f}(\chi_0)\|=\|f(\chi)-f(\chi_0)\| \leq L \|\chi-\chi_0\|$ for each $\chi_0$\\

  $\bar{f}$ is locally Lipschitz\\
  
  We can proceed down the same general method as in the previous question and get the resulting $\phi=project_x(\bar{\phi})$ and in the end $\bar{\phi}(t,\chi_0)$ is continuous satisfying initial condition and $\phi(t,\chi_0)$ is continuous satisying the initial condition.
  
  let $\bar{\phi}: \R \times \mathcal{X} \times \R^m \rightarrow \mathcal{X} \times \R^m$, where $\bar{\phi}(t, \chi_0)$ is a maximal solution satisfying initial condition $\chi_0$ via theorem 1.20\\

  use theorem 1.26 with $\bar{f}$ being locally Lipschitz and $\bar{\phi}(t, \chi_0)$ being a maximal solution so that W is open and $(t,\chi_0) \mapsto \bar{\phi}(t,\chi_0)$ is continuous.\\
  
  $\phi = project_x(\bar{\phi})$, where $project_x(\bar{\phi}): \mathcal{X} \times \R^m \rightarrow  \mathcal{X}$\\
  Then, $project_x((t,\chi_0) \mapsto \bar{\phi}(t,\chi_0))=\phi$ is continuous.
  
  \pagebreak
\item 1.13\\

  1)\\
  $t\in\R$ small enough that domain of definition of $\phi_t$ is non-empty.\\
  Determine domain and codomain of the map $\phi_t$.\\
  domain: ${x_0: (\forall x_0) (t',x_0) \in W \neq \O,\ \forall t' \in [0,t] \}|_{t\ fixed}$\\
  codomain: $\{\phi(t'): x(0)=x_0,\ \forall t' \in [0,t],\ x_0 \in d\}$\\

  injective:\\
  if x goes through $x_a$, $x_b$ such that $x_a\neq x_b$, then $\phi(0, x_a)$ = $\phi(t_c, x_b), t_c \neq 0$.\\
  $\phi(t, x_a)$ = $\phi(t,\phi(t_c,x_b))$\\
  $\phi(t, x_a)$ = $\phi(t+t_c,x_b)) \neq \phi(t, x_b)$.\\
  Then it is injective: $(\forall x_a, x_b) x_a \neq x_b \wedge \phi(t, x_a) \neq \phi(t, x_b)$\\

  surjective:\\
  let $\phi_t(x)=x_2$\\
  $\phi(0,x_2)=\phi(t,x)$\\
  $\phi(-t,x_2)=\phi(0,x)=x$\\
  Since the domain of definition of $\phi_t$ is non-empty, it guarantees that elements in codomain as atleast a corresponding element in the domain.\\
  $(\forall x_2=\{\phi(t'): x(0)=x_0,\ \forall t' \in [0,t],\ x_0 \in d\}) (\exists x_0)(x_0 \mapsto \phi_t(x_0)=x_2)$\\

  Then $\phi_t$ is injective and surjective, then $\phi_t$ is bijective.\\
  An inverse exists so that $y \mapsto \phi_t^{-1}(y) = x$:\\
  $\phi_t(x)=\phi(t,x)=y=\phi(0,y)$\\
  $\phi(0,x)=\phi(-t,y)$\\
  $\phi_t^{-1}(y)=\phi(-t,y)=x$\\
  
  \pagebreak
  
  2)\\
  let $(\exists T) T>0 \wedge x_1 \in \mathcal{X} \wedge x_1=\phi(T,x_1)$\\
  Show that:\\
  $(\forall t \in \R) \phi(t+T,x_1)=\phi(t,x_1), \text{system keeps looping around without possibly moving on to }x_2}$\\
  
  let $\phi(T_A,x_1)=x_2$\\
  let $T_A=T_1+T_2+...+T_n$, where $T_1,...,T_{n-1}=T$, $T_n=T_A\ mod\ T$\\

  $\phi(T_A,x_1)=\phi(T_1+T_2+...+T_n,x_1)$\\
  $\phi(T_A,x_1)=\phi(T_2+...+T_n,\phi(T_1,x_1))=\phi(T_2+...+T_n,x_1)$\\
  ...\\
  $\phi(T_A,x_1)=\phi(T_n,x_1)$\\
  Then, $\phi(T_n,x_1)$ is within cyclic path of $\phi(0,x_1)$ and $\phi(T,x_1)$. $x_2$ may not be reached for any T.\\
  
  3)\\
  let $(\forall x_1,x_2 \in \mathcal{X})(\exists t_1,t_2 \in \R)( x_1 \neq x_2 \wedge \phi(t_1,x_1)=\phi(t_2,x_2))$\\
  show $t_1 \neq t_2$\\
  
  if a curve ever connects $x_1$ with $x_2$, then: $\phi(0,x_1)=\phi(\epsilon,x_2), \epsilon \neq 0$\\
  $\phi(t_1,x_1)=\phi(t_1,\phi(\epsilon,x_2))=\phi(t_2,x_2))$\\
  
  if $t_1 = t_2$:\\
  $\phi(t,x_1)=\phi(t,\phi(\epsilon,x_2))=\phi(t,x_2))$\\
  $\phi(t,x_1)=\phi(t+\epsilon,x_2)=\phi(t,x_2))$\\
  it is not true that $\phi(t+\epsilon,x_2)=\phi(t,x_2)), \epsilon \neq 0$\\

  then, $(\forall x_1,x_2 \in \mathcal{X})(\exists t_1,t_2 \in \R)( x_1 \neq x_2 \wedge \phi(t_1,x_1)=\phi(t_2,x_2)) \rightarrow t_1 \neq t_2$\\

  consider $t_1 \neq t_2$, show flow carries $x_1$ to $x_2$ or vice versa, in positive time.\\

  let $t_1+\delta=t_2, \delta \neq 0$\\

  $\phi(t_1,x_1)=\phi(t_2,x_2)$\\
  $\phi(t_1,x_1)=\phi(t_1+\delta,x_2)$\\
  $\phi(0,x_1)=\phi(\delta,x_2)$\\
  $x_1=\phi(\delta,x_2)$\\
  $\delta>0: x_2 \rightarrow x_1$ via flow\\
  $\phi(-\delta,x_1)=\phi(x_2)$\\
  $\phi(-\delta,x_1)=x_2$\\
  $\delta<0: x_2=\phi(-\delta,x_1)=\phi(|\delta|,x_1), x_1 \rightarrow x_2$ via flow\\

  \pagebreak
  
\item 1.15\\

  part 2:\\

  $\phi(t,x)=\begin{bmatrix}e^t & 0 \\ 0 & e^{2t} \end{bmatrix} x$\\

  $W={(t,x_0) \in \R \times \mathcal{X}, t \in T_{x_0} }$ is open.\\

  $\phi(0,x)=\begin{bmatrix}1 & 0 \\ 0 & 1 \end{bmatrix} x = x$\\

  Consistency is satisfied.\\

  $\phi(s+t,x)=
  \begin{bmatrix}e^{s+t} & 0 \\ 0 & e^{2s+2t} \end{bmatrix}
  x$\\
  
  $\phi(s,\phi(t,x))=
  \begin{bmatrix}e^s & 0 \\ 0 & e^{2s} \end{bmatrix}
  \begin{bmatrix}e^t & 0 \\ 0 & e^{2t} \end{bmatrix}
  x=\begin{bmatrix}e^{s+t} & 0 \\ 0 & e^{2s+2t} \end{bmatrix} x = \phi(s+t,x)$\\

  Semigroup property is satisfied.\\

  Interval of maximal existence $T_{x_0}=(-\infty, \infty)$. Then $W=\R \times \mathcal{X}$ and $\phi(t,x)$ is a phase flow.\\

  vector field:\\

  $\frac{\partial}{\partial t} \phi(t,x) = f(x)|_{t}=
  \begin{bmatrix}
    e^{t} & 0\\
    0 & 2e^{2t}
  \end{bmatrix} x $\\
  
  part 6:\\
  
  $\phi(t,x)=e^{-t}\begin{bmatrix}cos(2t) & sin(2t) \\ -sin(2t) & cos(2t) \end{bmatrix} x$\\

  $W={(t,x_0) \in \R \times \mathcal{X}, t \in T_{x_0} }$ is open.\\
  
  $\phi(0,x)=\begin{bmatrix}1 & 0 \\ 0 & 1 \end{bmatrix} x = x$\\

  Consistency is satisfied.\\

  \pagebreak
  
  $\phi(s+t,x)=
  e^{-(s+t)}\begin{bmatrix}
    cos(2s+2t) & sin(2s+2t) \\
    -sin(2s+2t) & cos(2s+2t) \end{bmatrix}
  x$\\
  
  $\phi(s,\phi(t,x))=
  e^{-s}\begin{bmatrix}cos(2t) & sin(2t) \\ -sin(2t) & cos(2t) \end{bmatrix}
  e^{-t}\begin{bmatrix}cos(2s) & sin(2s) \\ -sin(2s) & cos(2s) \end{bmatrix}
  x $\\
  $\phi(s,\phi(t,x))=
  e^{-(s+t)}\begin{bmatrix}
    cos(2s)cost(2t)-sin(2s)sin(2t) & cos(2s)sin(2t)+sin(2s)cos(2t) \\
    -sin(2s)cos(2t)-cos(2s)sin(2t) & -sin(2s)sin(2t)+cos(2s)cos(2t) \end{bmatrix}
  x $\\

  $cos(a \pm b)=cos(a)cos(b) \mp sin(a)sin(b)$\\
  $sin(a \pm b)=sin(a)cos(b) \pm cos(a)sin(b)$\\

  $\phi(s,\phi(t,x))=e^{-(s+t)}\begin{bmatrix}
    cos(2s+2t) & sin(2s+2t) \\
    -sin(2s+2t) & cos(2s+2t) \end{bmatrix}
  x=\phi(s+t,x)$\\

  Semigroup property is satisfied.\\

  Interval of maximal existence $T_{x_0} = (-\infty, \infty)$. Then $W=\R \times \mathcal{X}$ and $\phi(t,x)$ is a phase flow.\\
  
  vector field:\\

  $\frac{\partial}{\partial t} \phi(t,x) = f(x)|_{t}=
  \begin{bmatrix}
    -e^{-t}cos(2t)-2e^{-t}sin(2t) & -e^{-t}sin(2t)+2e^{-t}cos(2t)\\
    e^{-t}sin(2t)-2e^{-t}cos(2t) & -e^{-t}cos(2t)-2e^{-t}sin(2t)
  \end{bmatrix} x $\\

  \pagebreak
  
\item 1.16\\

  1)\\
  $\phi(t,x)=x$\\
  $\phi(0,x)=x$\\
  Consistency is satisfied.\\
  
  $\phi(s+t,x)=x$\\
  $\phi(s,\phi(t,x))=x$\\
  Semigroup property is satisfied.\\

  $W={(t,x_0)\in \R \times \mathcal{X}}, t \in T_{x_0} = (-\infty, +\infty)$\\
  $W=\R \times \mathcal{X}}$\\
  $\phi(t,x)=x$ is a phase flow.\\

  2)\\
  $\phi(t,x)=tx$\\
  $(\forall x)\phi(0,x)=0 \neq x$\\
  Consistency is not satisfied.\\
  
  $\phi(s+t,x)=(s+t)x$\\
  $(\forall s,t) \phi(s,\phi(t,x))=t(sx) \neq \phi(s+t,x)$\\
  Semigroup property is not satisfied.\\

  $\phi(t,x)=tx$ is not a phase flow.\\

  3)\\
  $\phi(t,x)=(t+1)x$\\
  $\phi(0,x)=x$\\
  Consistency is satisfied.\\
  
  $\phi(s+t,x)=(s+t+1)x$\\
  $(\forall s,t) \phi(s,\phi(t,x))=(s+1)(t+1)x = (st+s+t+1)x \neq \phi(s+t,x)$\\
  Semigroup property is not satisfied.\\

  $\phi(t,x)=tx$ is not a phase flow.\\
  
  \pagebreak
  
  4)\\
  $\phi(t,x)=e^{t+1}x$\\
  $(\forall x)\phi(0,x)=e^1x \neq x$\\
  Consistency is not satisfied.\\
  
  $\phi(s+t,x)=e^{s+t+1}x$\\
  $\phi(s,\phi(t,x))= e^{s+1}e^{t+1}x=e^{s+t+2}x\neq \phi(s+t,x)$\\
  Semigroup property is not satisfied.\\

  $\phi(t,x)=e^{t+1}x$ is not a phase flow.\\

  5)\\
  $\phi(t,x)=e^tx$\\
  $(\forall x)\phi(0,x)=x$\\
  Consistency is satisfied.\\
  
  $\phi(s+t,x)=e^{s+t}x$\\
  $\phi(s,\phi(t,x))= e^{s}e^{t}x=e^{s+t}x = \phi(s+t,x)$\\
  Semigroup property is satisfied.\\

  $W={(t,x_0)\in \R \times \mathcal{X}}, t \in T_{x_0} = (-\infty, +\infty)$\\
  $W=\R \times \mathcal{X}}$\\
  $\phi(t,x)=e^tx$ is a phase flow.\\
  
  
\end{itemize}

\end {document}
